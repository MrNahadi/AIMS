\chapter{Introduction}
\label{chap:introduction}

\section{Background}
Marine diesel engines form the backbone of the global maritime industry, powering over 90\% of commercial vessels responsible for transporting global trade. The reliability of these engines is paramount; any failure can lead to severe operational disruptions, financial losses, and safety hazards. Traditional maintenance strategies, such as corrective (run-to-failure) and preventive (schedule-based) maintenance, often prove inefficient or costly. Corrective maintenance results in unpredictable downtime, while preventive maintenance can lead to unnecessary replacement of healthy components.

\section{Problem Statement}
Current engine monitoring systems largely rely on static threshold alarms. These systems trigger alerts only when a parameter exceeds a pre-defined limit. This approach has significant limitations:
\begin{enumerate}
    \item \textbf{High False Positive Rate:} Transient operational changes can trigger false alarms.
    \item \textbf{Lack of Prognostics:} Alarms occur after degradation has reached a critical level, offering little time for proactive intervention.
    \item \textbf{No Root Cause Analysis:} A generic ``high temperature'' alarm does not specify whether the cause is a cooling system failure, lubrication issue, or bearing wear.
\end{enumerate}
There is a critical need for a system that can not only predict faults before they occur but also explain the reasoning behind its predictions to engine operators.

\section{Objectives}
The primary objective of this project is to design and simulate an Explainable AI (XAI) based predictive maintenance system for marine diesel engines. Specific objectives include:
\begin{enumerate}
    \item To simulate a realistic dataset for a 4-cylinder marine diesel engine covering normal operation and eight specific fault conditions.
    \item To design a predictive model using the Light Gradient Boosting Machine (LightGBM) algorithm to classify engine health states.
    \item To implement SHAP (SHapley Additive exPlanations) to provide interpretable explanations for model predictions, identifying key contributing sensors.
    \item To develop a real-time engineering dashboard using React and FastAPI to visualize engine health and diagnostic insights.
\end{enumerate}

\section{Significance}
This research contributes to the maritime industry's transition towards ``Intelligent Vessels.'' By moving from reactive to predictive maintenance, ship operators can:
\begin{itemize}
    \item Reduce operational expenditure (OPEX) by optimizing maintenance schedules.
    \item Enhance safety by preventing catastrophic engine failures at sea.
    \item improve decision-making confidence through transparent, explainable AI diagnostics.
\end{itemize}

\section{Scope}
The scope of this project is limited to:
\begin{itemize}
    \item \textbf{Target System:} A simulated 4-cylinder marine diesel engine.
    \item \textbf{Fault Classes:} Normal operation and seven fault types including Fuel Injection Fault, Cooling System Fault, Turbocharger Fault, Bearing Wear, Lubrication Oil Degradation, Air Intake Restriction, and Vibration Anomaly.
    \item \textbf{Data:} Synthetic data generated to mimic real-world sensor behavior (18 parameters).
    \item \textbf{Verification:} Validation is performed via simulation scenarios; physical engine testing is outside the scope.
\end{itemize}
